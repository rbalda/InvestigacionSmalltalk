%!TEX TS-program = pdflatex
%!TEX encoding = UTF-8 Unicode

\documentclass[11pt]{article}

\usepackage[utf8]{inputenc}
\usepackage{geometry}
\geometry{a4paper}
\usepackage{graphicx}
\usepackage{booktabs}
\usepackage{array}
\usepackage{verbatim}
\usepackage{subfig}

\usepackage{fancyhdr} 
\pagestyle{fancy}
\renewcommand{\headrulewidth}{0pt} 
\lhead{}\chead{}\rhead{}
\lfoot{}\cfoot{\thepage}\rfoot{}

\usepackage{sectsty}
\allsectionsfont{\sffamily\mdseries\upshape}

\usepackage[nottoc,notlof,notlot]{tocbibind} 
\usepackage[titles,subfigure]{tocloft} 
\renewcommand{\cftsecfont}{\rmfamily\mdseries\upshape}
\renewcommand{\cftsecpagefont}{\rmfamily\mdseries\upshape}

\usepackage[spanish]{babel}
\usepackage{listings}

%%%El documento comienza aqui

\title{Investigacion de Lenguajes - Smalltalk}
\author{Rene Balda - Jimmy}
\date{\today}
\begin{document}
\maketitle
\section{Introducción} 
\paragraph{} \noindent
Smalltalk es un lenguaje de programación orientada a objeto, se basa en la comunicación entre objetos mediante envio de mensajes, está considerado como el primer lenguaje de este paradigma, ya que  en \textbf{Smalltalk} todo es un objeto; éste influyo drásticamente en el surgimiente de lenguajes como JAVA, PHP, Python, Ruby y muchos de los actuales lenguajes orientados a objetos.
\section{Características}
\section{Historia}
\section{Tutorial de Instalación}
\section{Hola Mundo y otros Programas Introductorios}
\end{document}